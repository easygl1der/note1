\documentclass[10pt]{article}

\usepackage{amsmath}
\usepackage{amssymb}
\usepackage{amsthm}
\usepackage{mathrsfs}
\usepackage{graphicx}
\usepackage{enumitem}
\usepackage{hyperref}

\theoremstyle{plain}
\newtheorem{theorem}{Theorem}[section]
\newtheorem{lemma}[theorem]{Lemma}
\newtheorem{proposition}[theorem]{Proposition}
\newtheorem{corollary}[theorem]{Corollary}

\theoremstyle{definition}
\newtheorem{definition}[theorem]{Definition}
\newtheorem{example}[theorem]{Example}

\theoremstyle{remark}
\newtheorem{remark}[theorem]{Remark}

\title{Notes on Partial Differential Equations}
\author{John K. Hunter}
\date{}

\begin{document}
\maketitle

\section{Preliminaries}
In this section, we collect various definitions and theorems for future use.

\subsection{Euclidean space}
Let $\mathbb{R}^{n}$ be $n$-dimensional Euclidean space. We denote the Euclidean norm of a vector $x=\left(x_{1}, x_{2}, \ldots, x_{n}\right) \in \mathbb{R}^{n}$ by
$$
|x|=\left(x_{1}^{2}+x_{2}^{2}+\cdots+x_{n}^{2}\right)^{1 / 2}
$$
and the inner product of vectors $x=\left(x_{1}, x_{2}, \ldots, x_{n}\right), y=\left(y_{1}, y_{2}, \ldots, y_{n}\right)$ by
$$
x \cdot y=x_{1} y_{1}+x_{2} y_{2}+\cdots+x_{n} y_{n}
$$

We denote Lebesgue measure on $\mathbb{R}^{n}$ by $d x$, and the Lebesgue measure of a set $E \subset \mathbb{R}^{n}$ by $|E|$.

If $E$ is a subset of $\mathbb{R}^{n}$, we denote the complement by $E^{c}=\mathbb{R}^{n} \backslash E$, the closure by $\bar{E}$, the interior by $E^{\circ}$ and the boundary by $\partial E=\bar{E} \backslash E^{\circ}$. The characteristic function $\chi_{E}: \mathbb{R}^{n} \rightarrow \mathbb{R}$ of $E$ is defined by
$$
\chi_{E}(x)= \begin{cases}1 & \text { if } x \in E \\ 0 & \text { if } x \notin E\end{cases}
$$

A set $E$ is bounded if $\{|x|: x \in E\}$ is bounded in $\mathbb{R}$. A set is connected if it is not the disjoint union of two nonempty relatively open subsets. We sometimes refer to a connected open set as a domain.

We say that a (nonempty) open set $\Omega^{\prime}$ is compactly contained in an open set $\Omega$, written $\Omega^{\prime} \Subset \Omega$, if $\overline{\Omega^{\prime}} \subset \Omega$ and $\overline{\Omega^{\prime}}$ is compact. If $\Omega^{\prime} \Subset \Omega$, then
$$
\operatorname{dist}\left(\Omega^{\prime}, \partial \Omega\right)=\inf \left\{|x-y|: x \in \Omega^{\prime}, y \in \partial \Omega\right\}>0
$$

\subsection{Spaces of continuous functions}
Let $\Omega$ be an open set in $\mathbb{R}^{n}$. We denote the space of continuous functions $u: \Omega \rightarrow \mathbb{R}$ by $C(\Omega)$; the space of functions with continuous partial derivatives in $\Omega$ of order less than or equal to $k \in \mathbb{N}$ by $C^{k}(\Omega)$; and the space of functions with continuous derivatives of all orders by $C^{\infty}(\Omega)$. Functions in these spaces need not be bounded even if $\Omega$ is bounded; for example, $(1/x) \in C^{\infty}(0,1)$.

If $\Omega$ is a bounded open set in $\mathbb{R}^{n}$, we denote by $C(\bar{\Omega})$ the space of continuous functions $u: \bar{\Omega} \rightarrow \mathbb{R}$. This is a Banach space with respect to the maximum, or supremum, norm
$$
\|u\|_{\infty}=\sup _{x \in \Omega}|u(x)| .
$$

We denote the support of a continuous function $u: \Omega \rightarrow \mathbb{R}^{n}$ by
$$
\operatorname{supp} u=\overline{\{x \in \Omega: u(x) \neq 0\}}
$$

We denote by $C_{c}(\Omega)$ the space of continuous functions whose support is compactly contained in $\Omega$, and by $C_{c}^{\infty}(\Omega)$ the space of functions with continuous derivatives of all orders and compact support in $\Omega$. We will sometimes refer to such functions as test functions.

The completion of $C_{c}\left(\mathbb{R}^{n}\right)$ with respect to the uniform norm is the space $C_{0}\left(\mathbb{R}^{n}\right)$ of continuous functions that approach zero at infinity.

If $\Omega$ is bounded, then we say that a function $u: \bar{\Omega} \rightarrow \mathbb{R}$ belongs to $C^{k}(\bar{\Omega})$ if it is continuous and its partial derivatives of order less than or equal to $k$ are uniformly continuous in $\Omega$, in which case they extend to continuous functions on $\bar{\Omega}$. The space $C^{k}(\bar{\Omega})$ is a Banach space with respect to the norm
$$
\|u\|_{C^{k}(\bar{\Omega})}=\sum_{|\alpha| \leq k} \sup _{\Omega}\left|\partial^{\alpha} u\right|
$$
where we use the multi-index notation for partial derivatives explained in Section 1.8. This norm is finite because the derivatives $\partial^{\alpha} u$ are continuous functions on the compact set $\bar{\Omega}$.

A vector field $X: \Omega \rightarrow \mathbb{R}^{m}$ belongs to $C^{k}(\bar{\Omega})$ if each of its components belongs to $C^{k}(\bar{\Omega})$.

\subsection{Hölder spaces}
The definition of continuity is not a quantitative one, because it does not say how rapidly the values $u(y)$ of a function approach its value $u(x)$ as $y \rightarrow x$. The modulus of continuity $\omega:[0, \infty] \rightarrow[0, \infty]$ of a general continuous function $u$, satisfying
$$
|u(x)-u(y)| \leq \omega(|x-y|)
$$
may decrease arbitrarily slowly. As a result, despite their simple and natural appearance, spaces of continuous functions are often not suitable for the analysis of PDEs, which is almost always based on quantitative estimates.

A straightforward and useful way to strengthen the definition of continuity is to require that the modulus of continuity is proportional to a power $|x-y|^{\alpha}$ for some exponent $0<\alpha \leq 1$. Such functions are said to be Hölder continuous, or Lipschitz continuous if $\alpha=1$. Roughly speaking, one can think of Hölder continuous functions with exponent $\alpha$ as functions with bounded fractional derivatives of the the order $\alpha$.

\begin{definition}
Suppose that $\Omega$ is an open set in $\mathbb{R}^{n}$ and $0<\alpha \leq 1$. A function $u: \Omega \rightarrow \mathbb{R}$ is uniformly Hölder continuous with exponent $\alpha$ in $\Omega$ if the quantity
\begin{equation}
[u]_{\alpha, \Omega}=\sup_{\substack{x, y \in \Omega \\ x \neq y}} \frac{|u(x)-u(y)|}{|x-y|^{\alpha}}
\end{equation}
is finite. A function $u: \Omega \rightarrow \mathbb{R}$ is locally uniformly Hölder continuous with exponent $\alpha$ in $\Omega$ if $[u]_{\alpha, \Omega^{\prime}}$ is finite for every $\Omega^{\prime} \Subset \Omega$. We denote by $C^{0, \alpha}(\Omega)$ the space of locally uniformly Hölder continuous functions with exponent $\alpha$ in $\Omega$. If $\Omega$ is bounded, we denote by $C^{0, \alpha}(\bar{\Omega})$ the space of uniformly Hölder continuous functions with exponent $\alpha$ in $\Omega$.
\end{definition}

We typically use Greek letters such as $\alpha, \beta$ both for Hölder exponents and multi-indices; it should be clear from the context which they denote.

When $\alpha$ and $\Omega$ are understood, we will abbreviate 'u is (locally) uniformly Hölder continuous with exponent $\alpha$ in $\Omega$' to 'u is (locally) Hölder continuous.' If $u$ is Hölder continuous with exponent one, then we say that $u$ is Lipschitz continuous. There is no purpose in considering Hölder continuous functions with exponent greater than one, since any such function is differentiable with zero derivative and therefore is constant.

The quantity $[u]_{\alpha, \Omega}$ is a semi-norm, but it is not a norm since it is zero for constant functions. The space $C^{0, \alpha}(\bar{\Omega})$, where $\Omega$ is bounded, is a Banach space with respect to the norm
$$
\|u\|_{C^{0, \alpha}(\bar{\Omega})}=\sup_{\Omega}|u|+[u]_{\alpha, \Omega}
$$

\begin{example}
For $0<\alpha<1$, define $u(x):(0,1) \rightarrow \mathbb{R}$ by $u(x)=|x|^{\alpha}$. Then $u \in C^{0, \alpha}([0,1])$, but $u \notin C^{0, \beta}([0,1])$ for $\alpha<\beta \leq 1$.
\end{example}

\begin{example}
The function $u(x):(-1,1) \rightarrow \mathbb{R}$ given by $u(x)=|x|$ is Lipschitz continuous, but not continuously differentiable. Thus, $u \in C^{0,1}([-1,1])$, but $u \notin C^{1}([-1,1])$.
\end{example}

We may also define spaces of continuously differentiable functions whose $k$th derivative is Hölder continuous.

\begin{definition}
If $\Omega$ is an open set in $\mathbb{R}^{n}, k \in \mathbb{N}$, and $0<\alpha \leq 1$, then $C^{k, \alpha}(\Omega)$ consists of all functions $u: \Omega \rightarrow \mathbb{R}$ with continuous partial derivatives in $\Omega$ of order less than or equal to $k$ whose $k$th partial derivatives are locally uniformly Hölder continuous with exponent $\alpha$ in $\Omega$. If the open set $\Omega$ is bounded, then $C^{k, \alpha}(\bar{\Omega})$ consists of functions with uniformly continuous partial derivatives in $\Omega$ of order less than or equal to $k$ whose $k$th partial derivatives are uniformly Hölder continuous with exponent $\alpha$ in $\Omega$.
\end{definition}

The space $C^{k, \alpha}(\bar{\Omega})$ is a Banach space with respect to the norm
$$
\|u\|_{C^{k, \alpha}(\bar{\Omega})}=\sum_{|\beta| \leq k} \sup_{\Omega}\left|\partial^{\beta} u\right|+\sum_{|\beta|=k}\left[\partial^{\beta} u\right]_{\alpha, \Omega}
$$

\section{$L^{p}$ spaces}
As before, let $\Omega$ be an open set in $\mathbb{R}^{n}$ (or, more generally, a Lebesgue-measurable set).

\begin{definition}
For $1 \leq p<\infty$, the space $L^{p}(\Omega)$ consists of the Lebesgue measurable functions $f: \Omega \rightarrow \mathbb{R}$ such that
$$
\int_{\Omega}|f|^{p} dx<\infty
$$
and $L^{\infty}(\Omega)$ consists of the essentially bounded functions.

These spaces are Banach spaces with respect to the norms
$$
\|f\|_{p}=\left(\int_{\Omega}|f|^{p} dx\right)^{1/p}, \quad\|f\|_{\infty}=\sup_{\Omega}|f|
$$
where sup denotes the essential supremum,
$$
\sup_{\Omega} f=\inf \{M \in \mathbb{R}: f \leq M \text{ almost everywhere in } \Omega\}.
$$
\end{definition}

Strictly speaking, elements of the Banach space $L^{p}$ are equivalence classes of functions that are equal almost everywhere, but we identify a function with its equivalence class unless we need to refer to the pointwise values of a specific representative. For example, we say that a function $f \in L^{p}(\Omega)$ is continuous if it is equal almost everywhere to a continuous function, and that it has compact support if it is equal almost everywhere to a function with compact support.

Next we summarize some fundamental inequalities for integrals, in addition to Minkowski's inequality which is implicit in the statement that $\|\cdot\|_{L^{p}}$ is a norm for $p \geq 1$. First, we recall the definition of a convex function.

\begin{definition}
A set $C \subset \mathbb{R}^{n}$ is convex if $\lambda x+(1-\lambda) y \in C$ for every $x, y \in C$ and every $\lambda \in[0,1]$. A function $\phi: C \rightarrow \mathbb{R}$ is convex if its domain $C$ is convex and
$$
\phi(\lambda x+(1-\lambda) y) \leq \lambda \phi(x)+(1-\lambda) \phi(y)
$$
for every $x, y \in C$ and every $\lambda \in[0,1]$.
\end{definition}

Jensen's inequality states that the value of a convex function at a mean is less than or equal to the mean of the values of the convex function.

\begin{theorem}
Suppose that $\phi: \mathbb{R} \rightarrow \mathbb{R}$ is a convex function, $\Omega$ is a set in $\mathbb{R}^{n}$ with finite Lebesgue measure, and $f \in L^{1}(\Omega)$. Then
$$
\phi\left(\frac{1}{|\Omega|} \int_{\Omega} f dx\right) \leq \frac{1}{|\Omega|} \int_{\Omega} \phi \circ f dx.
$$
\end{theorem}

To state the next inequality, we first define the Hölder conjugate of an exponent $p$. We denote it by $p^{\prime}$ to distinguish it from the Sobolev conjugate $p^{*}$ which we will introduce later on.

\begin{definition}
The Hölder conjugate of $p \in[1, \infty]$ is the quantity $p^{\prime} \in[1, \infty]$ such that
$$
\frac{1}{p}+\frac{1}{p^{\prime}}=1,
$$
with the convention that $1 / \infty=0$.
\end{definition}

The following result is called Hölder's inequality. The special case when $p=p^{\prime}=1/2$ is the Cauchy-Schwartz inequality.

\begin{theorem}
If $1 \leq p \leq \infty, f \in L^{p}(\Omega)$, and $g \in L^{p^{\prime}}(\Omega)$, then $f g \in L^{1}(\Omega)$ and
$$
\|f g\|_{1} \leq\|f\|_{p}\|g\|_{p^{\prime}}
$$
\end{theorem}

Repeated application of this inequality gives the following generalization.

\begin{theorem}
If $1 \leq p_{i} \leq \infty$ for $1 \leq i \leq N$ satisfy
$$
\sum_{i=1}^{N} \frac{1}{p_{i}}=1
$$
and $f_{i} \in L^{p_{i}}(\Omega)$ for $1 \leq i \leq N$, then $f=\prod_{i=1}^{N} f_{i} \in L^{1}(\Omega)$ and
$$
\|f\|_{1} \leq \prod_{i=1}^{N}\left\|f_{i}\right\|_{p_{i}}
$$
\end{theorem}

Suppose that $\Omega$ has finite measure and $1 \leq q \leq p$. If $f \in L^{p}(\Omega)$, an application of Hölder's inequality to $f=1 \cdot f$, shows that $f \in L^{q}(\Omega)$ and
$$
\|f\|_{q} \leq|\Omega|^{1/q-1/p}\|f\|_{p}
$$

Thus, the embedding $L^{p}(\Omega) \hookrightarrow L^{q}(\Omega)$ is continuous. This result is not true if the measure of $\Omega$ is infinite, but in general we have the following interpolation result.

\begin{lemma}
If $1 \leq p \leq q \leq r$, then $L^{p}(\Omega) \cap L^{r}(\Omega) \hookrightarrow L^{q}(\Omega)$ and
$$
\|f\|_{q} \leq\|f\|_{p}^{\theta}\|f\|_{r}^{1-\theta}
$$
where $0 \leq \theta \leq 1$ is given by
$$
\frac{1}{q}=\frac{\theta}{p}+\frac{1-\theta}{r}
$$
\end{lemma}

\begin{proof}
Assume without loss of generality that $f \geq 0$. Using Hölder's inequality with exponents $1 / \sigma$ and $1 /(1-\sigma)$, we get
$$
\int f^{q} dx=\int f^{\theta q} f^{(1-\theta) q} dx \leq\left(\int f^{\theta q / \sigma} dx\right)^{\sigma}\left(\int f^{(1-\theta) q /(1-\sigma)} dx\right)^{1-\sigma}
$$
Choosing $\sigma / \theta=q / p$, in which case $(1-\sigma) /(1-\theta)=q / r$, we get
$$
\int f^{q} dx \leq\left(\int f^{p} dx\right)^{q \theta / p}\left(\int f^{r} dx\right)^{q(1-\theta) / r}
$$
and the result follows.
\end{proof}

It is often useful to consider local $L^{p}$ spaces consisting of functions that have finite integral on compact sets.

\begin{definition}
The space $L_{\mathrm{loc}}^{p}(\Omega)$, where $1 \leq p \leq \infty$, consists of functions $f: \Omega \rightarrow \mathbb{R}$ such that $f \in L^{p}\left(\Omega^{\prime}\right)$ for every open set $\Omega^{\prime} \Subset \Omega$. A sequence of functions $\left\{f_{n}\right\}$ converges to $f$ in $L_{\mathrm{loc}}^{p}(\Omega)$ if $\left\{f_{n}\right\}$ converges to $f$ in $L^{p}\left(\Omega^{\prime}\right)$ for every open set $\Omega^{\prime} \Subset \Omega$.
\end{definition}

If $p<q$, then $L_{\mathrm{loc}}^{q}(\Omega) \hookrightarrow L_{\mathrm{loc}}^{p}(\Omega)$ even if the measure of $\Omega$ is infinite. Thus, $L_{\mathrm{loc}}^{1}(\Omega)$ is the 'largest' space of integrable functions on $\Omega$.

\begin{example}
Consider $f: \mathbb{R}^{n} \rightarrow \mathbb{R}$ defined by
$$
f(x)=\frac{1}{|x|^{a}}
$$
where $a \in \mathbb{R}$. Then $f \in L_{\text{loc}}^{1}\left(\mathbb{R}^{n}\right)$ if and only if $a<n$. To prove this, let
$$
f^{\epsilon}(x)= \begin{cases}f(x) & \text{ if }|x|>\epsilon \\ 0 & \text{ if }|x| \leq \epsilon\end{cases}
$$
Then $\left\{f^{\epsilon}\right\}$ is monotone increasing and converges pointwise almost everywhere to $f$ as $\epsilon \rightarrow 0^{+}$. For any $R>0$, the monotone convergence theorem implies that
$$
\begin{aligned}
\int_{B_{R}(0)} f dx & =\lim_{\epsilon \rightarrow 0^{+}} \int_{B_{R}(0)} f^{\epsilon} dx \\
& =\lim_{\epsilon \rightarrow 0^{+}} \int_{\epsilon}^{R} r^{n-a-1} dr \\
& = \begin{cases}\infty & \text{ if } n-a \leq 0 \\
(n-a)^{-1} R^{n-a} & \text{ if } n-a>0\end{cases}
\end{aligned}
$$
which proves the result. The function $f$ does not belong to $L^{p}\left(\mathbb{R}^{n}\right)$ for $1 \leq p<\infty$ for any value of $a$, since the integral of $f^{p}$ diverges at infinity whenever it converges at zero.
\end{example}

\section{References}
\begin{thebibliography}{99}
\bibitem{evans} L. C. Evans, Partial Differential Equations, Amer. Math. Soc., Providence, RI, 1993.
\bibitem{folland} G. Folland, Real Analysis: Modern Techniques and Applications, Wiley, New York, 1984.
\end{thebibliography}

\end{document}