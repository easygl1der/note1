\chapter{Root Systems - Chapter 3 Study Guide}

\section{Table of Contents}

\begin{enumerate}
	\item \href{#overview-and-motivation}{Overview and Motivation}
	\item \href{#section-9-axiomatics}{Section 9: Axiomatics}
	\item \href{#section-10-simple-roots-and-weyl-group}{Section 10: Simple Roots and Weyl Group}
	\item \href{#section-11-classification}{Section 11: Classification}
	\item \href{#section-12-construction-and-automorphisms}{Section 12: Construction and Automorphisms}
	\item \href{#section-13-abstract-theory-of-weights}{Section 13: Abstract Theory of Weights}
	\item \href{#key-relationships-and-dependencies}{Key Relationships and Dependencies}
\end{enumerate}

\section{Overview and Motivation}

Root systems are fundamental combinatorial-geometric objects that capture the essence of semisimple Lie algebras. They provide a way to understand the structure of these algebras through finite sets of vectors in Euclidean space that satisfy certain axioms.

\textbf{Main Goals of This Chapter:}

\begin{enumerate}
	\item Define root systems axiomatically
	\item Develop the theory of simple roots and Weyl groups
	\item Classify all possible root systems
	\item Show how to construct each type explicitly
	\item Develop the weight theory for representations
\end{enumerate}


\section{Section 9: Axiomatics}

\subsection{9.1 Reflections in Euclidean Space}

\subsubsection{Definition 9.1.1 (Euclidean Space)}

Throughout this chapter, $E$ is a \textbf{euclidean space}: a finite-dimensional vector space over $\mathbb{R}$ endowed with a positive definite symmetric bilinear form $(\alpha, \beta)$.

\subsubsection{Definition 9.1.2 (Reflection - Geometric)}

A \textbf{reflection} in $E$ is an invertible linear transformation that:

\begin{itemize}
	\item Leaves pointwise fixed some hyperplane (subspace of codimension one)
	\item Sends any vector orthogonal to that hyperplane to its negative
\end{itemize}

\textbf{Note}: A reflection is orthogonal (preserves the inner product on $E$).

\subsubsection{Definition 9.1.3 (Reflection \texorpdfstring{$\sigma_\alpha$}{sigma_alpha} - Algebraic)}

For any nonzero vector $\alpha \in E$, the reflection $\sigma_\alpha$ is defined by:
\[
\sigma_\alpha(\beta) = \beta - \frac{2(\beta, \alpha)}{(\alpha, \alpha)} \alpha
\]

The reflecting hyperplane is $P_\alpha = \{\beta \in E \mid (\beta, \alpha) = 0\}$.

\subsubsection{Notation}

We define $\langle\beta, \alpha\rangle = \frac{2(\beta, \alpha)}{(\alpha, \alpha)}$.

\textbf{Important}: $\langle\beta, \alpha\rangle$ is linear only in the first variable.

\subsubsection{Lemma 9.1.3}

Let $\Phi$ be a finite set which spans $E$. Suppose all reflections $\sigma_\alpha$ ($\alpha \in \Phi$) leave $\Phi$ invariant. If $\sigma \in GL(E)$ leaves $\Phi$ invariant, fixes pointwise a hyperplane $P$ of $E$, and sends some nonzero $\alpha \in \Phi$ to its negative, then $\sigma = \sigma_\alpha$ (and $P = P_\alpha$).

\textbf{Proof:}
Let $\tau = \sigma \sigma_\alpha$. Then:

\begin{itemize}
	\item $\tau(\Phi) = \Phi$
	\item $\tau(\alpha) = \alpha$
	\item $\tau$ acts as identity on $\mathbb{R}\alpha$ and on $E/\mathbb{R}\alpha$
\end{itemize}

This forces all eigenvalues of $\tau$ to be 1, so the minimal polynomial divides $(T-1)^\ell$.

Since $\Phi$ is finite, some power $\tau^k$ fixes all elements of $\Phi$. Since $\Phi$ spans $E$, we have $\tau^k = 1$, so the minimal polynomial divides $T^k - 1$.

Combining these conditions, the minimal polynomial is $T - 1$, hence $\tau = 1$. □

\subsection{9.2 Root Systems}

\subsubsection{Definition 9.2.1 (Root System)}

A subset $\Phi$ of the euclidean space $E$ is called a \textbf{root system} in $E$ if the following axioms are satisfied:

\textbf{(R1)} $\Phi$ is finite, spans $E$, and does not contain 0.

\textbf{(R2)} If $\alpha \in \Phi$, the only multiples of $\alpha$ in $\Phi$ are $\pm\alpha$.

\textbf{(R3)} If $\alpha \in \Phi$, the reflection $\sigma_\alpha$ leaves $\Phi$ invariant.

\textbf{(R4)} If $\alpha, \beta \in \Phi$, then $\langle\beta, \alpha\rangle \in \mathbb{Z}$.

\subsubsection{Definition 9.2.2 (Weyl Group)}

The \textbf{Weyl group} $\mathcal{W}$ of $\Phi$ is the subgroup of $GL(E)$ generated by the reflections $\sigma_\alpha$ ($\alpha \in \Phi$).

\subsubsection{Lemma 9.2.3}

Let $\Phi$ be a root system in $E$, with Weyl group $\mathcal{W}$. If $\sigma \in GL(E)$ leaves $\Phi$ invariant, then:

\begin{enumerate}
	\item $\sigma \sigma_\alpha \sigma^{-1} = \sigma_{\sigma(\alpha)}$ for all $\alpha \in \Phi$
	\item $\langle\beta, \alpha\rangle = \langle\sigma(\beta), \sigma(\alpha)\rangle$ for all $\alpha, \beta \in \Phi$
\end{enumerate}

\textbf{Proof:}
The key observation is that $\sigma \sigma_\alpha \sigma^{-1}$ acts as a reflection with respect to $\sigma(\alpha)$. By Lemma 9.1.3, this must equal $\sigma_{\sigma(\alpha)}$.

Computing explicitly:
\[
\sigma \sigma_\alpha \sigma^{-1}(\sigma(\beta)) = \sigma(\beta - \langle\beta, \alpha\rangle\alpha) = \sigma(\beta) - \langle\beta, \alpha\rangle\sigma(\alpha)
\]

This gives us the second assertion by comparing with the formula for $\sigma_{\sigma(\alpha)}$. □

\subsubsection{Definition 9.2.4 (Dual Root System)}

The \textbf{dual} (or inverse) of $\Phi$ is $\Phi^\vee = \{\alpha^\vee \mid \alpha \in \Phi\}$, where $\alpha^\vee = \frac{2\alpha}{(\alpha, \alpha)}$.

\subsection{9.3 Examples}

\subsubsection{Definition 9.3.1 (Rank of Root System)}

Call $\ell = \dim E$ the \textbf{rank} of the root system $\Phi$.

When $\ell \leq 2$, we can describe $\Phi$ by drawing pictures. In view of (R2), there is only one possibility when $\ell = 1$, labeled $(A_1)$.

\subsubsection{Rank 1: Type \texorpdfstring{$A_1$}{A_1}}
\[
\Phi = \{\alpha, -\alpha\}
\]
This is the unique root system of rank 1.

\subsubsection{Rank 2: Four Types}

\begin{enumerate}
	\item \textbf{Type $A_1 \times A_1$}: Two orthogonal copies of $A_1$
	\item \textbf{Type $A_2$}: $\Phi = \{\pm(\varepsilon_1 - \varepsilon_2), \pm(\varepsilon_2 - \varepsilon_3), \pm(\varepsilon_1 - \varepsilon_3)\}$
	\item \textbf{Type $B_2$}: Contains roots of two different lengths
	\item \textbf{Type $G_2$}: Contains roots of two different lengths with ratio $\sqrt{3}$
\end{enumerate}

\begin{figure}[H]
\centering
\includegraphics[width=\textwidth]{Root_Systems_Chapter3_Study_Guide-2025060612.png}
% \caption{}
\label{}
\end{figure}

\subsection{9.4 Pairs of Roots}

\subsubsection{Definition 9.4.1 (Angle Between Vectors)}

The \textbf{angle} $\theta$ between vectors $\alpha, \beta \in E$ is defined by the formula:
\[
\|\alpha\|\|\beta\| \cos \theta = (\alpha, \beta)
\]

Equivalently, $\cos \theta = \frac{(\alpha, \beta)}{\|\alpha\|\|\beta\|}$.

This gives us:
\[
\langle\beta, \alpha\rangle = \frac{2(\beta, \alpha)}{(\alpha, \alpha)} = 2 \frac{\|\beta\|}{\|\alpha\|} \cos \theta
\]

and
\[
\langle\alpha, \beta\rangle\langle\beta, \alpha\rangle = 4 \cos^2 \theta
\]

\subsubsection{Lemma 9.4.1 (Angle Restrictions)}

The possible values of $\langle\alpha, \beta\rangle\langle\beta, \alpha\rangle$ for nonproportional roots $\alpha, \beta$ are 0, 1, 2, or 3, corresponding to specific angles and length ratios.

\begin{table}[h]
	\centering
	\begin{tabular}{|c|c|c|c|}
		\hline
		$\langle\alpha, \beta\rangle$ & $\langle\beta, \alpha\rangle$ & $\|\beta\|^2/\|\alpha\|^2$ & $\theta$ \\
		\hline
		0 & 0 & undetermined & $\pi/2$ \\
		\hline
		1 & 1 & 1 & $\pi/3$ \\
		\hline
		-1 & -1 & 1 & $2\pi/3$ \\
		\hline
		1 & 2 & 2 & $\pi/4$ \\
		\hline
		-1 & -2 & 2 & $3\pi/4$ \\
		\hline
		1 & 3 & 3 & $\pi/6$ \\
		\hline
		-1 & -3 & 3 & $5\pi/6$ \\
		\hline
	\end{tabular}
\end{table}
\subsubsection{Lemma 9.4.2 (Root Addition)}

Let $\alpha, \beta$ be nonproportional roots. If $(\alpha, \beta) > 0$, then $\alpha - \beta$ is a root. If $(\alpha, \beta) < 0$, then $\alpha + \beta$ is a root.

\textbf{Proof:}
If $(\alpha, \beta) > 0$, then $\langle\alpha, \beta\rangle > 0$. From the table above, either $\langle\alpha, \beta\rangle = 1$ or $\langle\beta, \alpha\rangle = 1$.

If $\langle\alpha, \beta\rangle = 1$, then $\sigma_\beta(\alpha) = \alpha - \beta \in \Phi$ by (R3).
If $\langle\beta, \alpha\rangle = 1$, then $\beta - \alpha \in \Phi$, so $\sigma_{\beta-\alpha}(\beta-\alpha) = \alpha - \beta \in \Phi$. □

\subsubsection{Root Strings}

For nonproportional roots $\alpha, \beta$, the \textbf{$\alpha$-string through $\beta$} consists of all roots of the form $\beta + i\alpha$ ($i \in \mathbb{Z}$). This string is unbroken and has the form:
\[
\beta - r\alpha, \beta - (r-1)\alpha, \ldots, \beta, \ldots, \beta + q\alpha
\]
where $r - q = \langle\beta, \alpha\rangle$.


\section{Section 10: Simple Roots and Weyl Group}

\textbf{Throughout this section} $\Phi$ denotes a root system of rank $\ell$ in a euclidean space $E$, with Weyl group $\mathcal{W}$.

\subsection{10.1 Bases and Weyl Chambers}

\subsubsection{Definition 10.1.1 (Base/Simple Roots)}

A subset $\Delta$ of $\Phi$ is called a \textbf{base} if:

\textbf{(B1)} $\Delta$ is a basis of $E$

\textbf{(B2)} Each root $\beta \in \Phi$ can be written as $\beta = \sum_{\alpha \in \Delta} k_\alpha \alpha$ with integral coefficients $k_\alpha$ all nonnegative or all nonpositive.

The roots in $\Delta$ are called \textbf{simple roots}.

\subsubsection{Definition 10.1.2 (Positive/Negative Roots, Height)}

Relative to a base $\Delta$:

\begin{itemize}
	\item A root $\beta$ is \textbf{positive} (written $\beta \succ 0$) if all coefficients $k_\alpha \geq 0$
	\item A root $\beta$ is \textbf{negative} (written $\beta \prec 0$) if all coefficients $k_\alpha \leq 0$
	\item The \textbf{height} of $\beta$ is $\text{ht}(\beta) = \sum k_\alpha$
\end{itemize}

The collections of positive and negative roots are denoted $\Phi^+$ and $\Phi^-$ (clearly $\Phi^- = -\Phi^+$).

\subsubsection{Lemma 10.1.3}

If $\Delta$ is a base of $\Phi$, then $(\alpha, \beta) \leq 0$ for $\alpha \neq \beta$ in $\Delta$, and $\alpha - \beta$ is not a root.

\textbf{Proof:}
If $(\alpha, \beta) > 0$ for distinct simple roots $\alpha, \beta$, then by Lemma 9.4.2, $\alpha - \beta$ would be a root. But this contradicts (B2) since when we express $\alpha - \beta$ in terms of the base $\Delta$, we get $\alpha - \beta = 1 \cdot \alpha + (-1) \cdot \beta + 0 \cdot \gamma$ (for other simple roots $\gamma$), which has both positive coefficient $(+1)$ and negative coefficient $(-1)$. However, (B2) requires that all coefficients be either all nonnegative or all nonpositive. □

\subsubsection{Definition 10.1.4 (Weyl Chambers and Related Concepts)}

The hyperplanes $P_\alpha$ ($\alpha \in \Phi$) partition $E$ into finitely many regions. The connected components of $E - \bigcup_\alpha P_\alpha$ are called \textbf{Weyl chambers}.

For each vector $\gamma \in E$, define:
\[
\Phi^+(\gamma) = \{\alpha \in \Phi \mid (\gamma, \alpha) > 0\}
\]

A vector $\gamma \in E$ is \textbf{regular} if $\gamma \in E - \bigcup_{\alpha \in \Phi} P_\alpha$ and \textbf{singular} otherwise.

For regular $\gamma \in E$:

\begin{itemize}
	\item A root $\alpha \in \Phi^+(\gamma)$ is \textbf{decomposable} if $\alpha = \beta_1 + \beta_2$ for some $\beta_i \in \Phi^+(\gamma)$
	\item A root is \textbf{indecomposable} otherwise
	\item $\Delta(\gamma)$ = set of indecomposable roots in $\Phi^+(\gamma)$
\end{itemize}

\subsubsection{Theorem 10.1.5 (Existence of Bases)}

$\Phi$ has a base.

More precisely, for any regular $\gamma \in E$, the set $\Delta(\gamma)$ is a base of $\Phi$, and every base is obtainable in this manner.

\textbf{Proof Outline:}

\begin{enumerate}
	\item Show each positive root is a nonnegative $\mathbb{Z}$-combination of indecomposable roots
	\item Prove indecomposable roots have pairwise obtuse angles
	\item Establish linear independence of indecomposable roots
	\item Show this gives a base and that every base arises this way
\end{enumerate}

\subsection{10.2 Lemmas on Simple Roots}

\subsubsection{Lemma 10.2.1 (Lemma A)}

If $\alpha$ is positive but not simple, then $\alpha - \beta$ is a root (necessarily positive) for some $\beta \in \Delta$.

\subsubsection{Lemma 10.2.2 (Lemma B)}

Let $\alpha$ be simple. Then $\sigma_\alpha$ permutes the positive roots other than $\alpha$.

\textbf{Corollary:} Setting $\delta = \frac{1}{2}\sum_{\beta>0} \beta$, we have $\sigma_\alpha(\delta) = \delta - \alpha$ for all $\alpha \in \Delta$.

\subsubsection{Lemma 10.2.3 (Lemma C)}

Let $\alpha_1, \ldots, \alpha_t \in \Delta$. Write $\sigma_i = \sigma_{\alpha_i}$. If $\sigma_1 \cdots \sigma_{t-1}(\alpha_t)$ is negative, then for some index $1 \leq s < t$:
\[
\sigma_1 \cdots \sigma_t = \sigma_1 \cdots \sigma_{s-1} \sigma_{s+1} \cdots \sigma_{t-1}
\]

\textbf{Corollary:} If $\sigma = \sigma_1 \cdots \sigma_t$ is a minimal expression for $\sigma \in \mathcal{W}$, then $\sigma(\alpha_t) \prec 0$.

\subsection{10.3 The Weyl Group}

\subsubsection{Theorem 10.3.1 (Main Properties of Weyl Group)}

Let $\Delta$ be a base of $\Phi$.

\textbf{(a)} $\mathcal{W}$ acts transitively on Weyl chambers.

\textbf{(b)} $\mathcal{W}$ acts transitively on bases.

\textbf{(c)} If $\alpha$ is any root, there exists $\sigma \in \mathcal{W}$ such that $\sigma(\alpha) \in \Delta$.

\textbf{(d)} $\mathcal{W}$ is generated by the $\sigma_\alpha$ ($\alpha \in \Delta$).

\textbf{(e)} $\mathcal{W}$ acts simply transitively on bases.

\subsubsection{Definition 10.3.2 (Length Function)}

For $\sigma \in \mathcal{W}$, the \textbf{length} $\ell(\sigma)$ is the smallest number of simple reflections needed to express $\sigma$.

\subsubsection{Lemma 10.3.3 (Length Formula)}

For all $\sigma \in \mathcal{W}$, $\ell(\sigma) = n(\sigma)$, where $n(\sigma)$ is the number of positive roots $\alpha$ for which $\sigma(\alpha) \prec 0$.

\subsection{10.4 Irreducible Root Systems}

\subsubsection{Definition 10.4.1 (Irreducible Root System)}

$\Phi$ is \textbf{irreducible} if it cannot be partitioned into the union of two proper subsets such that each root in one set is orthogonal to each root in the other.

\subsubsection{Lemma 10.4.2}

$\Phi$ is irreducible if and only if $\Delta$ cannot be partitioned into orthogonal subsets.

\subsubsection{Lemma 10.4.3 (Unique Maximal Root)}

Let $\Phi$ be irreducible. There is a unique maximal root $\beta$ (relative to the partial ordering), and if $\beta = \sum k_\alpha \alpha$, then all $k_\alpha > 0$.

\subsubsection{Lemma 10.4.4 (Irreducible Action)}

Let $\Phi$ be irreducible. Then $\mathcal{W}$ acts irreducibly on $E$.

\subsubsection{Lemma 10.4.5 (At Most Two Root Lengths)}

Let $\Phi$ be irreducible. Then at most two root lengths occur in $\Phi$, and all roots of a given length are conjugate under $\mathcal{W}$.


\section{Section 11: Classification}

\subsection{11.1 Cartan Matrix}

\subsubsection{Definition 11.1.1 (Cartan Matrix)}

For a base $\Delta = \{\alpha_1, \ldots, \alpha_\ell\}$, the \textbf{Cartan matrix} is $(\langle\alpha_i, \alpha_j\rangle)$.

\subsubsection{Proposition 11.1.2}

The Cartan matrix determines $\Phi$ up to isomorphism.

\textbf{Proof:}
Given two root systems with the same Cartan matrix, there exists a unique vector space isomorphism preserving all Cartan integers, and this extends to an isomorphism of root systems. □

\subsection{11.2 Coxeter Graphs and Dynkin Diagrams}

\subsubsection{Definition 11.2.1 (Coxeter Graph)}

The \textbf{Coxeter graph} of $\Phi$ has $\ell$ vertices, with vertices $i$ and $j$ joined by $\langle\alpha_i, \alpha_j\rangle\langle\alpha_j, \alpha_i\rangle$ edges.

\subsubsection{Definition 11.2.2 (Dynkin Diagram)}

When multiple edges occur, add an arrow pointing to the shorter root. This \textbf{Dynkin diagram} completely determines the Cartan matrix.

\subsection{11.3 Irreducible Components}

\subsubsection{Proposition 11.3.1}

$\Phi$ decomposes uniquely as the union of irreducible root systems $\Phi_i$ in orthogonal subspaces $E_i$ such that $E = E_1 \oplus \cdots \oplus E_t$.

\subsection{11.4 Classification Theorem}

\subsubsection{Theorem 11.4.1 (Classification)}

If $\Phi$ is an irreducible root system of rank $\ell$, its Dynkin diagram is one of the following:

\begin{itemize}
	\item \textbf{Type $A_\ell$} ($\ell \geq 1$): $\circ - \circ - \cdots - \circ$
	\item \textbf{Type $B_\ell$} ($\ell \geq 2$): $\circ - \circ - \cdots - \circ \Rightarrow \circ$
	\item \textbf{Type $C_\ell$} ($\ell \geq 3$): $\circ - \circ - \cdots - \circ \Leftarrow \circ$
	\item \textbf{Type $D_\ell$} ($\ell \geq 4$): Branched diagram
	\item \textbf{Type $E_6, E_7, E_8$}: Exceptional cases
	\item \textbf{Type $F_4$}: $\circ - \circ \Rightarrow \circ - \circ$
	\item \textbf{Type $G_2$}: $\circ \gg \circ$ (triple bond)
\end{itemize}

\textbf{Proof Strategy:}

\begin{enumerate}
	\item Classify possible Coxeter graphs using geometric constraints
	\item Show these are the only connected graphs of "admissible" vectors
	\item Determine which Dynkin diagrams result from each Coxeter graph
\end{enumerate}


\section{Section 12: Construction and Automorphisms}

\subsection{12.1 Construction of Types A-G}

Each type can be explicitly constructed:

\subsubsection{Type \texorpdfstring{$A_\ell$}{A_ell}}

\begin{itemize}
	\item $E = $ subspace of $\mathbb{R}^{\ell+1}$ orthogonal to $\varepsilon_1 + \cdots + \varepsilon_{\ell+1}$
	\item $\Phi = \{\varepsilon_i - \varepsilon_j \mid i \neq j\}$
	\item Base: $\{\varepsilon_1 - \varepsilon_2, \varepsilon_2 - \varepsilon_3, \ldots, \varepsilon_\ell - \varepsilon_{\ell+1}\}$
\end{itemize}

\subsubsection{Type \texorpdfstring{$B_\ell$}{B_ell}}

\begin{itemize}
	\item $E = \mathbb{R}^\ell$
	\item $\Phi = \{\pm\varepsilon_i, \pm(\varepsilon_i \pm \varepsilon_j) \mid i \neq j\}$
	\item Base: $\{\varepsilon_1 - \varepsilon_2, \ldots, \varepsilon_{\ell-1} - \varepsilon_\ell, \varepsilon_\ell\}$
\end{itemize}

[Similar constructions for other types...]

\subsection{12.2 Automorphisms}

\subsubsection{Theorem 12.2.1}

$\text{Aut}(\Phi)$ is the semidirect product of $\mathcal{W}$ and $\Gamma$, where $\Gamma$ is the group of diagram automorphisms.

For irreducible $\Phi$:

\begin{itemize}
	\item $\Gamma = \mathbb{Z}/2\mathbb{Z}$ for types $A_\ell$ ($\ell \geq 2$), $D_\ell$ ($\ell > 4$), $E_6$
	\item $\Gamma = S_3$ for type $D_4$
	\item $\Gamma = 1$ for all other types
\end{itemize}


\section{Section 13: Abstract Theory of Weights}

\subsection{13.1 Weights}

\subsubsection{Definition 13.1.1 (Weight)}

An element $\lambda \in E$ is called a \textbf{weight} if $\langle\lambda, \alpha\rangle \in \mathbb{Z}$ for all $\alpha \in \Phi$.

\subsubsection{Definition 13.1.2 (Weight Lattice)}

The \textbf{weight lattice} is $\Lambda = \{\lambda \in E \mid \langle\lambda, \alpha\rangle \in \mathbb{Z} \text{ for all } \alpha \in \Phi\}$.

\subsubsection{Definition 13.1.3 (Root Lattice)}

The \textbf{root lattice} $\Lambda_r$ is the subgroup of $\Lambda$ generated by $\Phi$ (i.e., the $\mathbb{Z}$-span of any set of simple roots).

\subsubsection{Definition 13.1.4 (Dominant and Strongly Dominant Weights)}

Fix a base $\Delta \subset \Phi$. A weight $\lambda \in \Lambda$ is:

\begin{itemize}
	\item \textbf{dominant} if all integers $\langle\lambda, \alpha\rangle$ ($\alpha \in \Delta$) are nonnegative
	\item \textbf{strongly dominant} if these integers are positive
\end{itemize}

Let $\Lambda^+$ denote the set of all dominant weights.

\textbf{Geometric Interpretation}: $\Lambda^+$ is the set of all weights lying in the closure of the fundamental Weyl chamber $\mathfrak{C}(\Delta)$, while $\Lambda \cap \mathfrak{C}(\Delta)$ is the set of strongly dominant weights.

\subsubsection{Definition 13.1.5 (Fundamental Dominant Weights)}

If $\Delta = \{\alpha_1, \ldots, \alpha_\ell\}$, the \textbf{fundamental dominant weights} $\lambda_1, \ldots, \lambda_\ell$ are the dual basis to $\{2\alpha_i/(\alpha_i, \alpha_i)\}$ (relative to the inner product on $E$):
\[
\frac{2(\lambda_i, \alpha_j)}{(\alpha_j, \alpha_j)} = \delta_{ij}
\]

\textbf{Key Property}: $\Lambda$ is a lattice with basis $\{\lambda_i \mid 1 \leq i \leq \ell\}$, and $\lambda \in \Lambda^+$ if and only if $\lambda = \sum m_i \lambda_i$ with all $m_i \geq 0$.

\subsubsection{Definition 13.1.6 (Fundamental Group)}

The quotient $\Lambda/\Lambda_r$ is finite and called the \textbf{fundamental group} of the root system. Its order equals the determinant of the Cartan matrix.

\subsection{13.2 Dominant Weights}

\subsubsection{Definition 13.2.1 (Partial Order on Weights)}

For $\lambda, \mu \in \Lambda$, write $\mu \prec \lambda$ if $\lambda - \mu$ is a sum of positive roots (or equals zero).

\subsection{13.3 The Weight \texorpdfstring{$\delta$}{delta}}

\subsubsection{Definition 13.3.1 (The Weight δ)}

Define $\delta = \frac{1}{2}\sum_{\alpha>0} \alpha = \sum_{i=1}^\ell \lambda_i$.

\textbf{Key Properties}:

\begin{itemize}
	\item $\delta$ is strongly dominant
	\item $\sigma_i(\delta) = \delta - \alpha_i$ for all simple reflections $\sigma_i$
\end{itemize}

\subsection{13.4 Saturated Sets of Weights}

\subsubsection{Definition 13.4.1 (Saturated Set)}

A subset $\Pi \subset \Lambda$ is \textbf{saturated} if for all $\lambda \in \Pi$, $\alpha \in \Phi$, and $0 \leq i \leq \langle\lambda, \alpha\rangle$, the weight $\lambda - i\alpha$ also lies in $\Pi$.

\textbf{Note}: Any saturated set is automatically stable under the Weyl group $\mathcal{W}$.

\subsubsection{Definition 13.4.2 (Highest Weight)}

A saturated set $\Pi$ has \textbf{highest weight} $\lambda$ ($\lambda \in \Lambda^+$) if $\lambda \in \Pi$ and $\mu \prec \lambda$ for all $\mu \in \Pi$.

\subsubsection{Definition 13.4.3 (Minimal Weight)}

A weight $\lambda \in \Lambda^+$ is \textbf{minimal} if $\mu \in \Lambda^+$ and $\mu \prec \lambda$ implies $\mu = \lambda$.

\textbf{Characterization}: $\lambda$ is minimal if and only if the $\mathcal{W}$-orbit of $\lambda$ is saturated (with highest weight $\lambda$), if and only if $\lambda \in \Lambda^+$ and $\langle\lambda, \alpha\rangle \in \{-1, 0, 1\}$ for all roots $\alpha$.


\section{Key Relationships and Dependencies}

\subsection{Logical Flow of Main Results}

\begin{enumerate}
	\item \textbf{Axioms (R1)-(R4)} → \textbf{Basic properties of root systems}
	\item \textbf{Reflection geometry} → \textbf{Weyl group structure}
	\item \textbf{Base existence} → \textbf{Positive/negative root decomposition}
	\item \textbf{Simple root properties} → \textbf{Weyl group generation}
	\item \textbf{Geometric constraints} → \textbf{Classification theorem}
	\item \textbf{Explicit constructions} → \textbf{Verification of all types exist}
	\item \textbf{Weight lattice structure} → \textbf{Representation theory foundations}
\end{enumerate}

\subsection{Critical Dependencies}

\begin{itemize}
	\item \textbf{Lemma 9.4.2} (root addition) is crucial for understanding root strings
	\item \textbf{Theorem 10.1.5} (base existence) underlies all subsequent theory
	\item \textbf{Theorem 10.3.1} (Weyl group properties) is essential for classification
	\item \textbf{Theorem 11.4.1} (classification) is the culmination of the geometric analysis
	\item \textbf{Section 13} provides the foundation for representation theory
\end{itemize}

\subsection{Key Proof Techniques}

\begin{enumerate}
	\item \textbf{Geometric arguments} using angles and inner products
	\item \textbf{Induction on height} for properties of positive roots
	\item \textbf{Minimality arguments} for Weyl group elements
	\item \textbf{Lattice theory} for weight spaces
	\item \textbf{Graph theory} for classification of Dynkin diagrams
\end{enumerate}

This completes the comprehensive study guide for Chapter 3 on Root Systems. Each concept builds systematically on previous ones, culminating in the complete classification and the foundation for representation theory.
